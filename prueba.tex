%%%%%%%%%%%%%%%%%%%%%%%%%%%%%%%%%%%%%%%%%%%%%%%%%%%%%%%%%%%%%%%%%%%%%%%%%%%%%%%
% Este es el CÓDIGO LATEX de un documento PDF simulado.
% Título: Documentación Técnica - Sistema de Gestión FINDEPRO (v1.1 ORIGINAL)
% Simula la respuesta de la empresa al Punto 2 de la solicitud de auditoría.
% ESTÁ DISEÑADO PARA CONTENER HALLAZGOS DE AUDITORÍA CRÍTICOS Y EXTENSOS.
%%%%%%%%%%%%%%%%%%%%%%%%%%%%%%%%%%%%%%%%%%%%%%%%%%%%%%%%%%%%%%%%%%%%%%%%%%%%%%%

\documentclass[a4paper, 11pt, landscape]{article} % Página horizontal para tablas

% --- PAQUETES ---
\usepackage[utf8]{inputenc}
\usepackage[spanish]{babel} 
\usepackage[margin=2cm]{geometry} % Márgenes
\usepackage{graphicx}
\usepackage{fancyhdr} % Cabeceras y pies de página
\usepackage{booktabs} % Tablas profesionales
\usepackage[table]{xcolor} % Colores en tablas
\usepackage{lastpage} % Contar páginas "Página X de Y"
\usepackage{parskip} % Espacio entre párrafos
\usepackage{array} % Para mejor control de columnas
\usepackage{longtable} % Para tablas que ocupan varias páginas

% --- CONFIGURACIÓN DE CABECERA Y PIE DE PÁGINA (Branding FINDEPRO) ---
\pagestyle{fancy}
\fancyhf{} 
\lhead{FINDEPRO - Documentación Técnica Interna}
\rhead{\textbf{CONFIDENCIAL}}
\cfoot{Página \thepage\ de \pageref{LastPage}}
\rfoot{\textcolor{red}{\textbf{ESTADO: DESACTUALIZADO (2018)}}}
\renewcommand{\headrulewidth}{0.4pt}
\renewcommand{\footrulewidth}{0.4pt}

% --- INICIO DEL DOCUMENTO ---
\begin{document}
	
	% --- PORTADA ---
	\thispagestyle{empty}
	
		\begin{center}
			\vspace*{1cm}
			{\Huge \textbf{FINDEPRO}} \\
			\vspace*{0.2cm}
			{\large Institución Financiera de Desarrollo}
			
			\vspace*{3cm}
			
			{\LARGE \textbf{Documentación Técnica}} \\
			{\huge \textbf{Sistema de Gestión FINDEPRO (SGF)}}
			
			\vspace*{0.5cm}
			
			{\Large Diccionario de Datos y Modelo Lógico}
			
			\vspace{2cm}
			
			{\Large \textcolor{red}{\textbf{Versión: 1.1 (Diseño Original)}}}
			
			\vspace*{3cm}
			
			\begin{tabular}{ll}
				\textbf{Fecha de Creación:} & 10 de enero de 2018 \\
				\textbf{Última Revisión:} & 20 de marzo de 2019 (Notas por J.R.) \\
				\textbf{Propietario:} & Gerencia de Tecnología \\
			\end{tabular}
			
			\vfill
			
			{\small \textit{Este documento describe el diseño original de la BD. La implementación productiva puede diferir. Las notas en cursiva fueron añadidas por J. Rodriguez (ex-DBA) en 2019.}}
		\end{center}

	\newpage
	
	% --- HISTORIAL DE REVISIONES ---
	\section*{1. Historial de Revisiones}
	\begin{tabular}{lllp{10cm}}
		\toprule
		\textbf{Versión} & \textbf{Fecha} & \textbf{Autor} & \textbf{Descripción del Cambio} \\
		\midrule
		1.0 & 10/01/2018 & DevTeam & Creación del documento base. Diseño inicial. \\
		1.1 & 20/03/2019 & J.R. (DBA) & \textbf{Revisión de estado.} Añadidas notas de hallazgos y riesgos. \textit{No se aplicaron cambios en la BD.} \\
		\bottomrule
	\end{tabular}
	\textbf{Hallazgo (Auditor):} \textit{La documentación no se actualiza formalmente desde 2018. Las revisiones de 2019 solo identificaron problemas que no fueron resueltos.}
	
	% --- SGBD INFO ---
	\section*{2. Información del SGBD}
	\begin{itemize}
		\item \textbf{Sistema Gestor:} Microsoft SQL Server 2016
		\item \textbf{Edición:} Standard (SP2) - \textbf{Versión:} 13.0.5102.1
		\item \textbf{Servidor (Hostname):} SRV-DB-01.findepro.local
		\item \textbf{Base de Datos Principal:} FINDEPRO\_PROD
		\item \textbf{Nota (J.R. 2019):} \textit{Versión 2016 está próxima a salir de soporte estándar. Se debe planificar la migración a SQL 2019. (Hallazgo: Esto nunca se hizo).}
	\end{itemize}
	
	% --- MODELO E-R ---
	\section*{3. Modelo Lógico (Conceptual 2018)}
	\begin{figure}[h!]
		\centering
		\fbox{
			\parbox[c][15cm][c]{0.9\textwidth}{
				\centering
				\vspace{5cm}
				{\Large \textbf{IMAGEN NO DISPONIBLE}} \\
				\vspace{0.5cm}
				{\large (Simulación de \texttt{Modelo\_Logico\_Conceptual.png})} \\
				\vspace{0.5cm}
				\small El diagrama mostraría relaciones "limpias" (1 a N) entre:
				\small \texttt{Agencias -> Socios}, \texttt{Socios -> Creditos}, \texttt{Creditos -> Desembolsos}, \texttt{Creditos -> Pagos}, \texttt{Roles -> UsuariosDelSistema}.
				\small \textbf{Estas relaciones (FKs) NO existen en la implementación real.}
			}
		}
		\caption{Modelo E-R Conceptual (2018). \textbf{No usar como referencia física.}}
	\end{figure}
	\newpage
	
	% --- DICCIONARIO DE DATOS ---
	\section*{4. Diccionario de Datos (Extenso)}
	
	\begin{longtable}{p{2.5cm} p{2.5cm} p{1cm} p{14cm}}
		\caption{Diccionario de Datos - Tablas Principales} \label{tab:diccionario} \\
		\toprule
		\rowcolor{gray!40}
		\textbf{Columna} & \textbf{Tipo de Dato} & \textbf{Nulo} & \textbf{Descripción / Reglas de Negocio} \\
		\midrule
		\endfirsthead
		\multicolumn{4}{c}%
		{{\bfseries \tablename\ \thetable{} -- continuación}} \\
		\toprule
		\rowcolor{gray!40}
		\textbf{Columna} & \textbf{Tipo de Dato} & \textbf{Nulo} & \textbf{Descripción / Reglas de Negocio} \\
		\midrule
		\endhead
		\bottomrule
		\multicolumn{4}{r}{{Continúa en la siguiente página...}} \\
		\endfoot
		\bottomrule
		\endlastfoot
		
		% --- Tabla Agencias ---
		\multicolumn{4}{l}{\textbf{Tabla: \texttt{[dbo].[Agencias]}}} \\
		\multicolumn{4}{l}{\textit{Descripción: Maestro de agencias y sucursales de FINDEPRO.}} \\
		\midrule
		\texttt{ID\_Agencia} & \texttt{int} (PK) & No & Llave Primaria. \\
		\texttt{NombreAgencia} & \texttt{varchar(100)} & No & Nombre oficial (Ej: "Agencia Central", "Sucursal El Alto"). \\
		\texttt{Direccion} & \texttt{varchar(250)} & Sí & \\
		\texttt{Telefono} & \texttt{varchar(50)} & Sí & \textit{Nota (J.R.): Debería ser \texttt{varchar(20)}. Campo sobredimensionado.} \\
		\texttt{ID\_Region} & \texttt{int} & Sí & \textit{Nota (J.R.): Sin FK a una tabla de Regiones. Dato no confiable.} \\
		\midrule
		
		% --- Tabla Socios ---
		\multicolumn{4}{l}{\textbf{Tabla: \texttt{[dbo].[Socios]}}} \\
		\multicolumn{4}{l}{\textit{Descripción: Almacena la información de los socios/clientes. \textbf{ALERTA: En producción se llama \texttt{[dbo].[Clientes]}.}}} \\
		\midrule
		\texttt{ID\_Socio} & \texttt{int} (PK) & No & Llave Primaria. \\
		\texttt{Nombres} & \texttt{varchar(100)} & No & \\
		\texttt{Apellidos} & \texttt{varchar(100)} & No & \\
		\texttt{CI} & \texttt{varchar(20)} & No & \textbf{Dato Sensible (PII). Almacenado en TEXTO PLANO.} \\
		\texttt{FechaNacimiento} & \texttt{date} & Sí & \\
		\texttt{Direccion} & \texttt{varchar(255)} & Sí & \textbf{Dato Sensible (PII). Almacenado en TEXTO PLANO.} \\
		\texttt{Telefono} & \texttt{varchar(30)} & Sí & \\
		\texttt{Email} & \texttt{varchar(100)} & Sí & \\
		\texttt{SueldoMensual} & \texttt{decimal(10,2)} & Sí & \textbf{Dato Sensible (Financiero). Almacenado en TEXTO PLANO.} \\
		\texttt{Ciudad} & \texttt{varchar(50)} & Sí & \textbf{Problema de Calidad:} 'La Paz', 'LP', 'L.P.', 'LaPaz', 'El Alto', 'E.A.'. \\
		\texttt{ID\_Agencia} & \texttt{int} & No & \textbf{FK LÓGICA} a \texttt{[dbo].[Agencias]}. \textbf{No existe constraint de FK.} \\
		\midrule
		
		% --- Tabla Creditos ---
		\multicolumn{4}{l}{\textbf{Tabla: \texttt{[dbo].[Creditos]}}} \\
		\multicolumn{4}{l}{\textit{Descripción: Maestro de créditos aprobados.}} \\
		\midrule
		\texttt{ID\_Credito} & \texttt{int} (PK) & No & Llave Primaria. \\
		\texttt{ID\_Socio} & \texttt{int} & No & \textbf{FK LÓGICA} a \texttt{[dbo].[Socios]}. \textbf{No existe constraint de FK.} \textit{Nota (J.R.): Se encontraron 12 créditos sin socio válido (huérfanos) en 2019.} \\
		\texttt{Monto\_Aprobado} & \texttt{decimal(18,2)} & No & Monto en Bolivianos (BOB). \\
		\texttt{Tipo\_Credito} & \texttt{int} & No & \textit{Nota (J.R.): Es una FK Lógica a la tabla \texttt{[TiposCredito]}, ¡pero esta tabla NUNCA SE CREÓ! La aplicación usa un ENUM (1=Productivo, 2=Consumo). ¡Riesgo de integridad!} \\
		\texttt{Estado} & \texttt{char(3)} & Sí & (Ej: 'APR', 'DES', 'CAN', 'MRA'). \\
		\texttt{Fecha\_Aprobacion} & \texttt{datetime} & Sí & \\
		\texttt{Usuario\_Aprobador} & \texttt{varchar(50)} & Sí & Login del \texttt{[UsuariosDelSistema]}. \\
		\midrule
		
		% --- Tabla Desembolsos ---
		\multicolumn{4}{l}{\textbf{Tabla: \texttt{[dbo].[Desembolsos]}}} \\
		\multicolumn{4}{l}{\textit{Descripción: Registra los desembolsos de dinero asociados a un crédito.}} \\
		\midrule
		\texttt{ID\_Desembolso} & \texttt{bigint} (PK) & No & Llave Primaria. \\
		\texttt{ID\_Credito} & \texttt{int} & No & \textbf{FK LÓGICA} a \texttt{[dbo].[Creditos]}. \textbf{No existe constraint de FK.} \textit{Nota (J.R.): Causa principal de descuadres contables.} \\
		\texttt{Monto\_Desembolsado} & \texttt{decimal(18,2)} & No & Monto en BOB. \\
		\texttt{Fecha\_Desembolso} & \texttt{datetime} & No & Fecha y hora del desembolso. \\
		\texttt{ID\_Usuario\_Caja} & \texttt{int} & No & FK (Lógica) a \texttt{[UsuariosDelSistema]}. \\
		\midrule
		
		% --- Tabla Pagos ---
		\multicolumn{4}{l}{\textbf{Tabla: \texttt{[dbo].[Pagos]}}} \\
		\multicolumn{4}{l}{\textit{Descripción: Registra los pagos (cuotas) que realizan los socios.}} \\
		\midrule
		\texttt{ID\_Pago} & \texttt{bigint} (PK) & No & Llave Primaria. \\
		\texttt{ID\_Credito} & \texttt{int} & No & \textbf{FK LÓGICA} a \texttt{[dbo].[Creditos]}. \textbf{No existe constraint de FK.} \\
		\texttt{Nro\_Cuota} & \texttt{int} & No & \\
		\texttt{Monto\_Pagado} & \texttt{decimal(18,2)} & No & \\
		\texttt{Fecha\_Pago} & \texttt{datetime} & No & \\
		\texttt{ID\_Usuario\_Caja} & \texttt{int} & No & FK (Lógica) a \texttt{[UsuariosDelSistema]}. \\
		\midrule
		
		% --- Tabla Roles ---
		\multicolumn{4}{l}{\textbf{Tabla: \texttt{[dbo].[Roles]}}} \\
		\multicolumn{4}{l}{\textit{Descripción: Roles de la aplicación SGF.}} \\
		\midrule
		\texttt{ID\_Rol} & \texttt{int} (PK) & No & Llave Primaria. \\
		\texttt{NombreRol} & \texttt{varchar(50)} & No & (Ej: 'Administrador', 'Oficial de Credito', 'Cajero'). \\
		\texttt{Descripcion} & \texttt{varchar(200)} & Sí & \\
		\midrule
		
		% --- Tabla UsuariosDelSistema ---
		\multicolumn{4}{l}{\textbf{Tabla: \texttt{[dbo].[UsuariosDelSistema]}}} \\
		\multicolumn{4}{l}{\textit{Descripción: Usuarios que acceden al Sistema de Gestión FINDEPRO (SGF).}} \\
		\midrule
		\texttt{ID\_Usuario} & \texttt{int} (PK) & No & Llave Primaria. \\
		\texttt{Login} & \texttt{varchar(50)} & No & Nombre de usuario (ej: cguachalla). \\
		\textbf{\texttt{Password}} & \textbf{\texttt{varchar(50)}} & \textbf{No} & \textbf{¡¡¡RIESGO CRÍTICO (J.R. 2019)!!! LA CONTRASEÑA SE ALMACENA EN TEXTO PLANO.} \\
		\texttt{NombreCompleto} & \texttt{varchar(200)} & No & \\
		\texttt{ID\_Rol} & \texttt{int} & No & \textbf{FK LÓGICA} a \texttt{[dbo].[Roles]}. \textbf{No existe constraint de FK.} \\
		\texttt{ID\_Agencia} & \texttt{int} & No & \textbf{FK LÓGICA} a \texttt{[dbo].[Agencias]}. \textbf{No existe constraint de FK.} \\
		\texttt{Activo} & \texttt{bit} & No & 1 = Activo, 0 = Inactivo. \\
		\texttt{Fecha\_Ultimo\_Login} & \texttt{datetime} & Sí & \\
		\texttt{Email} & \texttt{varchar(100)} & Sí & \\
		\midrule
		
		% --- Tabla Obsoleta ---
		\multicolumn{4}{l}{\textbf{Tabla: \texttt{[dbo].[Socios\_BKP\_2017]}}} \\
		\multicolumn{4}{l}{\textit{Descripción: Tabla obsoleta. Backup de migración inicial.}} \\
		\midrule
		\texttt{ID\_Socio\_OLD} & \texttt{varchar(10)} & Sí & \\
		\texttt{Nombre\_Comp} & \texttt{varchar(200)} & Sí & \\
		\multicolumn{4}{l}{\textit{Nota (J.R.): Tabla obsoleta. Ocupa 500MB. Riesgo de seguridad (datos de socios sin control de acceso). Pendiente de eliminación. No borrar por si acaso.}} \\
		\bottomrule
		
	\end{longtable}
\end{document}