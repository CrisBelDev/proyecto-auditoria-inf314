%%%%%%%%%%%%%%%%%%%%%%%%%%%%%%%%%%%%%%%%%%%%%%%%%%%%%%%%%%%%%%%%%%%%%%%%%%%%%%%
% DOCUMENTACIÓN TÉCNICA OFICIAL - FINDEPRO IFD
% TIPO: CONFIDENCIAL - NIVEL 3
% ID DOCUMENTO: TI-DOC-2025-089
%%%%%%%%%%%%%%%%%%%%%%%%%%%%%%%%%%%%%%%%%%%%%%%%%%%%%%%%%%%%%%%%%%%%%%%%%%%%%%%

\documentclass[a4paper, 11pt]{article}

% --- PAQUETES ESENCIALES ---
\usepackage[utf8]{inputenc}
\usepackage[spanish]{babel}
\usepackage[margin=2.5cm, top=3cm, bottom=3cm]{geometry}
\usepackage{graphicx}
\usepackage{fancyhdr}
\usepackage{booktabs}
\usepackage[table]{xcolor}
\usepackage{lastpage}
\usepackage{parskip}
\usepackage{array}
\usepackage{longtable}
\usepackage{listings} % Para mostrar código SQL
\usepackage{draftwatermark} % Marca de agua

% --- CONFIGURACIÓN DE MARCA DE AGUA ---
\SetWatermarkText{CONFIDENCIAL}
\SetWatermarkScale{0.8}
\SetWatermarkColor[gray]{0.9}

% --- CONFIGURACIÓN DE CÓDIGO SQL ---
\lstset{
	language=SQL,
	basicstyle=\footnotesize\ttfamily,
	keywordstyle=\color{blue}\bfseries,
	commentstyle=\color{gray}\itshape,
	frame=single,
	breaklines=true,
	backgroundcolor=\color{gray!10}
}

% --- CABECERA Y PIE ---
\setlength{\headheight}{40pt}
\pagestyle{fancy}
\fancyhf{}
\lhead{\textbf{FINDEPRO }\\Gerencia de Tecnología}
\rhead{Ref: TI-DOC-2025-089\\}
\cfoot{\thepage\ de \pageref{LastPage}}
\rfoot{\textit{Impreso por: SYSTEM\_ADMIN}}
\renewcommand{\headrulewidth}{1pt}

\begin{document}
	
	% ---------------------------------------------------------
	% PÁGINA 1: MEMORÁNDUM DE ENTREGA (EL TOQUE GERENCIAL)
	% ---------------------------------------------------------
	\thispagestyle{empty}
	
	\begin{flushright}
		\textbf{MEMORÁNDUM: GT-2025-104} \\
		La Paz, 3 de octubre de 2025
	\end{flushright}
	
	\vspace{1cm}
	
	\begin{tabular}{ll}
		\textbf{DE:} & Ing. Carlos Méndez - Gerente Nacional de Tecnología \\
		\textbf{PARA:} & Lic. Auditoría Interna \\
		\textbf{REF:} & \textbf{ENTREGA DE DICCIONARIO DE DATOS (REQ. AUDITORÍA)}
	\end{tabular}
	
	\vspace{0.5cm}
	\hrule
	\vspace{0.5cm}
	
	Estimados,
	
	En atención al Requerimiento de Auditoría \texttt{AUD-2025-003} referente a la "Validación de Integridad de la Cartera", adjunto a la presente la documentación técnica actualizada de la tabla maestra de operaciones (\texttt{Operaciones\_Cartera}).
	
	\textbf{Consideraciones Importantes para el Auditor:}
	\begin{enumerate}
		\item La extracción de datos (\textit{Excel}) que se les facilitó proviene directamente del servidor \texttt{SRV-DB-01}.
		\item Como se discutió en la reunión de apertura, existen registros que corresponden a la cartera comprada (recalificada) que aún no ha sido migrada completamente al esquema 'FINDEPRO'. Esto \textbf{no es un error}, es una limitación conocida del Core Bancario.
		\item El campo \textit{Saldo Ejecución} incluye intereses penales calculados manualmente por el área legal, por lo que puede diferir de la contabilidad oficial en centavos.
	\end{enumerate}
	
	Sin otro particular,
	
	\vspace{2cm}
	
	\begin{center}
		\rule{7cm}{0.5pt} \\
		Ing. Carlos Méndez \\
		Gerente de Tecnología \\
		FINDEPRO IFD
	\end{center}
	
	\newpage
	
	% ---------------------------------------------------------
	% PÁGINA 2: PORTADA TÉCNICA Y CONTROL DE CAMBIOS
	% ---------------------------------------------------------
	
	\begin{center}
		\vspace*{1cm}
		{\Huge \textbf{DICCIONARIO DE DATOS}} \\
		\vspace{0.5cm}
		{\LARGE SISTEMA INTEGRADO DE CARTERA (SIC)}
		
		\vspace{2cm}
		
		\begin{table}[h!]
			\centering
			\renewcommand{\arraystretch}{1.5}
			\begin{tabular}{|p{4cm}|p{8cm}|}
				\hline
				\textbf{Base de Datos Origen:} & \texttt{FINDEPRO\_PROD} (SQL Server 2016) \\
				\hline
				\textbf{Tabla Principal:} & \texttt{[dbo].[Operaciones\_Cartera]} \\
				\hline
				\textbf{Nivel de Acceso:} & CONFIDENCIAL / AUDITORES EXTERNOS \\
				\hline
				\textbf{Responsable Técnico:} & J. Rodriguez (DBA Senior) \\
				\hline
			\end{tabular}
		\end{table}
		
		\vspace{1cm}
		\textbf{CONTROL DE VERSIONES}
		\begin{table}[h!]
			\centering
			\small
			\begin{tabular}{llll}
				\toprule
				\textbf{Versión} & \textbf{Fecha} & \textbf{Autor} & \textbf{Descripción} \\
				\midrule
				1.0 & 10/01/2018 & DevTeam & Diseño original post-migración COBOL. \\
				1.5 & 20/03/2019 & J. Rodriguez & Inclusión de campos de mora (ASFI). \\
				\textbf{2.0} & \textbf{18/11/2025} & \textbf{Gerencia TI} & \textbf{Actualización para Auditoría Externa.} \\
				\bottomrule
			\end{tabular}
		\end{table}
	\end{center}
	
	\newpage
	
	% ---------------------------------------------------------
	% PÁGINA 3: DICCIONARIO DE DATOS (EL TABLE REAL)
	% ---------------------------------------------------------
	\section{Estructura Física de Datos}
	
	A continuación se detalla el esquema de la tabla utilizada para generar los reportes regulatorios.
	
	\begin{longtable}{p{4cm} p{2.5cm} p{1cm} p{8cm}}
		\caption{Tabla: \texttt{[dbo].[Operaciones\_Cartera]}} \label{tab:diccionario} \\
		\toprule
		\rowcolor{gray!40}
		\textbf{Campo (Columna)} & \textbf{Tipo Dato} & \textbf{Null} & \textbf{Descripción Técnica} \\
		\midrule
		\endfirsthead
		
		\multicolumn{4}{l}{\small \textit{...viene de página anterior}} \\
		\toprule
		\rowcolor{gray!40}
		\textbf{Campo} & \textbf{Tipo} & \textbf{Null} & \textbf{Descripción} \\
		\midrule
		\endhead
		
		\bottomrule
		\multicolumn{4}{r}{\small \textit{Continúa en siguiente página...}} \\
		\endfoot
		\bottomrule
		\endlastfoot
		
		% --- IDENTIFICACION ---
		\multicolumn{4}{l}{\cellcolor{blue!10}\textbf{1. SEGMENTO DE IDENTIFICACIÓN (CLIENTE)}} \\
		\texttt{TIPO\_DOCUMENTO} & \texttt{smallint} & No & FK interna. 1001=Cédula Identidad. \\
		\texttt{NUMERO\_DOCUMENTO} & \texttt{varchar(20)} & No & CI. \textbf{Nota:} Sin validación de formato. \\
		\texttt{NOMBRE\_COMPLETO} & \texttt{varchar(200)} & No & Concatenado (ApellidoP + ApellidoM + Nombres). \\
		\texttt{EXTENSION} & \texttt{char(2)} & Sí & Expedido en (LP, SC, CB, etc). \\
		\texttt{FECHA\_NACIMIENTO} & \texttt{date} & \textbf{Sí} & \textit{Incidencia:} 5\% de nulos en migración 2018. \\
		\texttt{ACTIVIDAD\_ECON} & \texttt{varchar(10)} & Sí & Código CAEDEC (Ej: 01111). A veces contiene decimales por error de Excel. \\
		
		% --- OPERACION ---
		\multicolumn{4}{l}{\cellcolor{blue!10}\textbf{2. DETALLE DE LA OPERACIÓN}} \\
		\texttt{SIGLA\_ENTIDAD} & \texttt{varchar(20)} & No & Valores: 'FINDEPRO' (Propia) o 'FAAA' (Recalificada). \\
		\texttt{NUMERO\_OPERACION} & \texttt{varchar(30)} & No & PK. Formato: \texttt{Suc-Prod-Sub-Of-Correlativo}. \\
		\texttt{TIPO\_OBLIGADO} & \texttt{varchar(5)} & No & \texttt{1A} (Titular), \texttt{2} (Garante). \\
		\texttt{TIPO\_CREDITO} & \texttt{varchar(5)} & No & \texttt{M0} (Micro), \texttt{H0} (Vivienda), \texttt{N0} (Consumo). \\
		\texttt{MONEDA} & \texttt{char(3)} & No & \texttt{MN} (Bolivianos), \texttt{ME} (Dólares). \\
		\texttt{FECHA\_INICIO} & \texttt{date} & No & Fecha desembolso. \\
		\texttt{FECHA\_VENCIMIENTO} & \texttt{date} & No & Fecha última cuota pactada. \\
		
		% --- SALDOS ---
		\multicolumn{4}{l}{\cellcolor{red!10}\textbf{3. SALDOS Y ESTADO (CRÍTICO PARA PROVISIONES)}} \\
		\texttt{SALDO\_VIGENTE} & \texttt{decimal(18,2)} & No & Capital a tiempo. \\
		\texttt{SALDO\_VENCIDO} & \texttt{decimal(18,2)} & No & Mora administrativa ($<=$90 días). \\
		\texttt{SALDO\_EJECUCION} & \texttt{decimal(18,2)} & No & \textbf{Mora Judicial.} Total actual $>$ 8 Millones. \\
		\texttt{SALDO\_CASTIGO} & \texttt{decimal(18,2)} & No & Cartera saneada (Cuenta de Orden). \\
		\texttt{DIAS\_MORA} & \texttt{int} & No & Calculado: \texttt{FechaCierre - FechaProxCuota}. \\
		\texttt{CALIFICACION} & \texttt{char(1)} & No & Riesgo (A, B, C, D, E, F, H). \\
		
	\end{longtable}
	
	\newpage
	
	% ---------------------------------------------------------
	% PÁGINA 4: ANEXO TÉCNICO (LA PRUEBA DE FUEGO)
	% ---------------------------------------------------------
	\section{Anexo Técnico: Vista de Extracción}
	
	Para transparencia de la auditoría, se adjunta el script T-SQL utilizado para generar el archivo \texttt{.csv} entregado. Nótese la lógica de conversión en el campo \texttt{SIGLA\_ENTIDAD}.
	
	\begin{lstlisting}[caption=Script de Generación de Datos (View)]
		CREATE VIEW [Reportes].[vw_Auditoria_Cartera_2025]
		AS
		SELECT 
		C.TipoDoc AS TIPO_DOCUMENTO,
		C.NumDoc AS NUMERO_DOCUMENTO,
		UPPER(C.Nombre + ' ' + C.Paterno + ' ' + C.Materno) AS NOMBRE_COMPLETO,
		-- Lógica para cartera comprada FAAA
		CASE 
		WHEN O.IdOrigen = 99 THEN 'FAAA' 
		ELSE 'FINDEPRO' 
		END AS SIGLA_ENTIDAD,
		O.OperacionCode AS NUMERO_OPERACION,
		O.MontoOriginal AS MONTO_ORIGINAL,
		-- Cálculos de saldos
		ISNULL(S.SaldoCapital, 0) AS SALDO_VIGENTE,
		ISNULL(S.Mora, 0) AS SALDO_VENCIDO,
		ISNULL(S.Ejecucion, 0) AS SALDO_EJECUCION, -- Revisar con Legal
		DATEDIFF(day, S.FechaProxPago, GETDATE()) AS DIAS_MORA
		FROM [dbo].[Clientes] C
		INNER JOIN [dbo].[Operaciones] O ON C.IdCliente = O.IdCliente
		LEFT JOIN [dbo].[SaldosDiarios] S ON O.IdOperacion = S.IdOperacion
		WHERE O.Estado NOT IN ('CANCELADO', 'ANULADO')
	\end{lstlisting}
	
	\section*{Notas Finales del DBA}
	\begin{itemize}
		\item \textbf{Duplicidad en Garantes:} Al realizar `SELECT *`, las operaciones aparecen duplicadas si el crédito tiene garantes (Tipo Obligado '2'). Para cuadrar con Contabilidad, filtrar por \texttt{TIPO\_OBLIGADO = '1A'}.
		\item \textbf{Saldo Ejecución MN:} Se recomienda revisar manualmente las operaciones con \texttt{DIAS\_MORA > 500} y Saldo > 50,000 BOB, ya que representan el 80\% del riesgo.
	\end{itemize}
	
\end{document}