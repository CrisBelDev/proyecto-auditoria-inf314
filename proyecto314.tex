\documentclass[letterpaper,12pt]{article}

% --- PAQUETES ---
\usepackage[utf8]{inputenc}
\usepackage[spanish]{babel}
\usepackage{geometry}
\geometry{left=2.5cm, right=2.5cm, top=3cm, bottom=3cm}
\usepackage{setspace}
\usepackage[colorlinks=true, linkcolor=black, urlcolor=black, citecolor=black]{hyperref}
\usepackage{titlesec}
\usepackage{graphicx}
\usepackage{booktabs}
\usepackage{fancyhdr}
\usepackage{amsmath}
\usepackage{newtxtext,newtxmath} % Fuente Times New Roman


% --- CONFIGURACIÓN DE ENCABEZADOS Y PIES ---
\pagestyle{fancy}
\fancyhf{}
\lhead{Auditoría de Base de Datos - FINDEPRO}
\rhead{Avance del Proyecto}
\rfoot{Página \thepage}
\renewcommand{\headrulewidth}{0.4pt}
\renewcommand{\footrulewidth}{0.4pt}

% --- CONFIGURACIÓN de TÍTULOS ---
\titleformat{\section}
{\normalfont\fontsize{14}{16}\bfseries}{\thesection}{1em}{}

\titleformat{\subsection}
{\normalfont\fontsize{14}{16}\bfseries}{\thesubsection}{1em}{}

% --- INICIO DEL DOCUMENTO ---
\begin{document}
	
	% --- PORTADA ---
	\begin{titlepage}
		\centering
		\setstretch{1.3}
		\vspace*{1cm}
		
		{\fontsize{16}{18}\selectfont\bfseries UNIVERSIDAD MAYOR DE SAN ANDRÉS \par}
		\vspace{0.2cm}
		{\fontsize{16}{18}\selectfont\bfseries FACULTAD DE CIENCIAS PURAS Y NATURALES \par}
		{\fontsize{16}{18}\selectfont\bfseries CARRERA DE INFORMÁTICA \par}
		
		\vspace{2cm}
		
		{\fontsize{16}{18}\selectfont\bfseries AUDITORÍA DE BASE DE DATOS \par}
		\vspace{0.3cm}
		{\fontsize{16}{18}\selectfont\bfseries INFORME DE AVANCE \par}
		\vspace{0.3cm}
		{\fontsize{16}{18}\selectfont\bfseries Sistema de Gestión FINDEPRO \par}
		
		\vspace{2.5cm}
		
		{\fontsize{16}{18}\selectfont\bfseries Integrantes: \par}
		\vspace{0.2cm}
		{\fontsize{16}{18}\selectfont
			Cristian Abel Mamani Mamani \par
			Carmen Reyna Candia Suñagua \par
			Gonzalo Rafael Esprella Coaquira \par
			Limbert Nelson Escobar Mamani \par
			Limberth Carbajal Colque \par
			Greis Estefani Gonzales Chávez \par
		}
		
		\vspace{1cm}
		
		{\fontsize{16}{18}\selectfont\bfseries Asignatura: Auditoría Informática (INF-314) \par}
		{\fontsize{16}{18}\selectfont\bfseries Docente: M. Sc. Miguel Cotaña Mier \par}
		
		\vfill
		
		{\fontsize{16}{18}\selectfont La Paz - Bolivia \\ Octubre, 2025 \par}
	\end{titlepage}
	% --- FIN DE PORTADA ---
	
	
	
	\tableofcontents % Añade tabla de contenido
	\newpage
	\onehalfspacing % Interlineado 1.5
	
	% --- SECCIONES DEL INFORME ---
	
	\section{Identificación de la Entidad y Sistema a Auditar}
	
	\subsection{La Entidad}
	\begin{itemize}
		\item \textbf{Nombre Principal:} FINDEPRO (Entidades Financieras de Desarrollo para Productores).
		\item \textbf{Tipo de Entidad:} Institución Financiera de Desarrollo (IFD).
		\item \textbf{Descripción Breve:} FINDEPRO se enfoca en proveer servicios financieros al sector productivo.
	\end{itemize}
	
	\subsection{Sistema a Auditar}
	\begin{itemize}
		\item \textbf{Nombre/Descripción:} Base de Datos del Sistema de Gestión.
		\item \textbf{Propósito del Sistema (inferido):} Gestionar información de clientes/socios (datos personales, productivos), operaciones financieras (créditos, desembolsos), seguimiento de proyectos, y generación de reportes operativos y de transparencia.
		\item \textbf{Tecnología de Base de Datos (a confirmar):} [Indicar si se conoce, ej. PostgreSQL, SQL Server, Oracle, MySQL].
	\end{itemize}
	
	\section{Descripción General de Funcionalidades}
	
	\begin{itemize}
		\item Registro y gestión de clientes/socios.
		\item Administración de créditos y cartera (FINDEPRO).
		\item Generación de informes financieros y de producción.
		\item Gestión de usuarios y permisos dentro del sistema.
	\end{itemize}
	
	
	\section{Objetivos y Alcance de la Auditoría}
	
	\subsection{Objetivo General}
	Evaluar integralmente la gestión, seguridad y rendimiento de la base de datos, identificando riesgos, verificando la efectividad de los controles implementados y proponiendo recomendaciones para fortalecer la protección de los datos y optimizar su uso, considerando las necesidades operativas de FINDEPRO.
	
	\subsection{Objetivos Específicos}
	\begin{itemize}
		\item Evaluar la \textbf{seguridad de acceso} a la base de datos, incluyendo la gestión de usuarios, roles, permisos y políticas de autenticación.
		
		\item Verificar la \textbf{integridad y consistencia} de los datos almacenados, analizando la estructura del esquema, el uso de constraints y la calidad de los datos.
		
		\item Validar las medidas de \textbf{confidencialidad} aplicadas a datos sensibles, como el cifrado en reposo y en tránsito.
		
		\item Examinar los procedimientos de \textbf{respaldo y recuperación} (Backup \& Restore) para asegurar la disponibilidad de la información ante contingencias.
		
		\item Analizar los mecanismos de \textbf{auditoría y monitoreo} de la base de datos para garantizar la trazabilidad y detección de actividades anómalas.
		
		\item Evaluar el \textbf{rendimiento} general de la base de datos e identificar posibles cuellos de botella u optimizaciones en consultas críticas.
		
		\item Verificar el \textbf{cumplimiento} con políticas internas de FINDEPRO y normativas externas relevantes.
	\end{itemize}
	
	\subsection{Alcance}
	La auditoría se centrará exclusivamente en la base de datos principal que soportan el sistema, incluyendo:
	\begin{itemize}
		\item El servidor de base de datos (Configuración, versionamiento, parches).
		\item El esquema de la base de datos (Tablas, vistas, funciones, procedimientos almacenados, triggers relacionados con FINDEPRO).
		\item Los mecanismos de seguridad implementados (Roles, permisos, cifrado).
		\item Los procesos de respaldo y recuperación documentados.
		\item Los logs de auditoría y rendimiento disponibles.
	\end{itemize}
	No se incluye en el alcance la auditoría de la aplicación cliente, la infraestructura de red (más allá de la conexión a la BD) o la seguridad física del centro de datos, excepto en lo que impacte directamente a la base de datos.
	
	\section{Objeto de la Auditoría} % Nueva sección
	El objeto específico de esta auditoría es la \textbf{base de datos} del sistema. Esto incluye el análisis de su diseño lógico y físico, los controles de seguridad implementados a nivel del SGBD, los procedimientos de administración (backup, monitoreo, gestión de usuarios), la integridad de los datos almacenados y el cumplimiento de las políticas y normativas aplicables a la información gestionada por FINDEPRO.
	
	\section{Metodología y Criterios de Auditoría}
	
	\subsection{Enfoque Metodológico}
	Se aplicará un enfoque basado en riesgos, alineado con marcos de referencia internacionales como \textbf{COBIT 2019} (especialmente los dominios APO y DSS) e \textbf{ISO/IEC 27001/27002}. Se combinarán técnicas cualitativas (entrevistas, revisión documental) y cuantitativas (análisis de logs, ejecución de scripts SQL). Se adopta una perspectiva crítica orientada a la mejora continua.
	
	\subsection{Dominio COBIT Aplicado (Preliminar)}
	Se tomará como referencia principal el dominio \textbf{DSS (Deliver, Service and Support)} de COBIT 2019, con énfasis en:
	\begin{itemize}
		\item \textbf{DSS05 Gestionar los Servicios de Seguridad:} Relevante para evaluar la protección de la información en la base de datos.
		\item \textbf{DSS06 Gestionar los Controles del Proceso de Negocio:} Para asegurar que los controles de la base de datos soportan la integridad de los procesos de FINDEPRO.
	\end{itemize}
	Esto se complementará con controles específicos de ISO 27002.
	
	\subsection{Modalidad y Niveles de Investigación}
	La auditoría será \textbf{mixta}, combinando:
	\begin{itemize}
		\item \textbf{Investigación Documental:} Análisis de políticas, manuales, diagramas, etc.
		\item \textbf{Investigación de Campo:} Entrevistas, observación, pruebas técnicas directas sobre la BD (con acceso de lectura).
	\end{itemize}
	Se abordarán los niveles \textbf{Exploratorio} (identificación inicial), \textbf{Descriptivo} (documentación del estado actual) y \textbf{Explicativo} (análisis de causas y efectos).
	
	\subsection{Población y Muestra}
	\begin{itemize}
		\item \textbf{Población:} Componentes de la BD, procesos dependientes (registro de clientes, créditos, etc.), usuarios (DBAs, desarrolladores, usuarios finales clave).
		\item \textbf{Muestra:} Se seleccionarán muestras representativas de tablas críticas, configuraciones de seguridad, logs de auditoría y usuarios para un análisis detallado. La selección será basada en riesgos.
	\end{itemize}
	
	\subsection{Técnicas e Instrumentos (Planeados)}
	\begin{itemize}
		\item \textbf{Revisión Documental:} Checklists basados en COBIT/ISO.
		\item \textbf{Entrevistas:} Cuestionarios estructurados para DBAs y personal relevante.
		\item \textbf{Observación y Pruebas Técnicas:} Scripts SQL para verificar permisos, constraints, configuraciones; revisión de logs; uso de herramientas de administración del SGBD.
	\end{itemize}
	
	\subsection{Procesamiento de la Información}
	Los datos recopilados serán clasificados, analizados (descriptiva y críticamente) y contrastados con los criterios de auditoría para identificar hallazgos, determinar causas raíz y formular recomendaciones.
	\subsection{Normativa Aplicable}
	La auditoría tomará como referencia principal las siguientes normativas y marcos:
	
	\begin{itemize}
		\item \textbf{ISO/IEC 27001:2022 Gestión de la Seguridad de la Información.}
		\item \textbf{ISO/IEC 27002:2022 Controles de Seguridad de la Información.}
		\item \textbf{COBIT 2019 - Gobierno y Gestión de Tecnologías de Información.}
		\item \textbf{ISO 9001:2015 - Gestión de la Calidad.}
		\item \textbf{ISO/IEC 20000-1:2018 Gestión de Servicios de TI.}
		\item \textbf{Ley N° 164 General de Telecomunicaciones, Tecnologías de Información y Comunicación (Bolivia):} Potencialmente aplicable a la protección de datos. (Se debe verificar si existe una ley específica de protección de datos más reciente o si se aplica esta de forma supletoria).
		\item \textbf{Regulaciones de la ASFI (Autoridad de Supervisión del Sistema Financiero):} Aplicables a FINDEPRO como IFD, especialmente en cuanto a seguridad de la información y continuidad del negocio.
		\item \textbf{Políticas Internas de FINDEPRO.}
		\item \textbf{NIST Cybersecurity Framework (Referencia complementaria).}
	\end{itemize}
	
	
	\subsection{Criterios Específicos de Auditoría}
	Los criterios concretos contra los cuales se evaluará la base de datos incluyen:
	\begin{itemize}
		\item Políticas internas de seguridad de FINDEPRO.
		\item Buenas prácticas para el SGBD específico (ej. CIS Benchmarks).
		\item Controles relevantes de ISO/IEC 27002 y COBIT DSS05/DSS06.
		\item Requerimientos normativos aplicables (ASFI, Protección de Datos).
	\end{itemize}
	
	\section{Estado Actual del Proyecto}
	
	A la fecha, el equipo auditor ha completado las siguientes actividades:
	\begin{itemize}
		\item \textbf{Fase de Planificación Concluida:} Se han definido y documentado los objetivos, el alcance detallado, la metodología, los criterios de auditoría y el cronograma general del proyecto.
		\item \textbf{Solicitud Formal de Recursos:} Se ha enviado la comunicación oficial a FINDEPRO solicitando:
		\begin{itemize}
			\item Acceso de solo lectura a la base de datos del sistema.
			\item Documentación técnica relevante (Modelo E-R, Diccionario de Datos, Políticas de Seguridad y Backup).
			\item Disponibilidad para entrevistas con el personal técnico responsable (DBA, Equipo de Desarrollo/Soporte).
		\end{itemize}
		\textit{Estado actual: Pendiente de recepción de credenciales y documentación.}
		\item \textbf{Preparación de Instrumentos:} Se han elaborado cuestionarios iniciales y checklists basados en COBIT/ISO y regulaciones ASFI (preliminar). Se han preparado scripts SQL genéricos para verificaciones iniciales.
		\item \textbf{Matriz de Riesgos Preliminar:} Se ha desarrollado una matriz identificando riesgos potenciales específicos para el contexto financiero/productivo.
	\end{itemize}
	
	\section{Plan de Pruebas Detallado}
	% --- (Esta sección se mantiene igual que la versión anterior) ---
	\subsection{Gestión de Accesos y Seguridad}
	\begin{itemize}
		\item Inventario y análisis de cuentas de usuario y roles definidos en la BD.
		\item Verificación de la asignación del principio de mínimo privilegio.
		\item Revisión de la configuración de autenticación y políticas de contraseñas.
		\item Análisis de permisos sobre objetos críticos.
	\end{itemize}
	\subsection{Integridad y Consistencia de Datos}
	\begin{itemize}
		\item Comprobación de claves primarias y foráneas.
		\item Verificación de constraints (UNIQUE, CHECK).
		\item Búsqueda de registros huérfanos o inconsistentes.
		\item Análisis de la estructura de normalización.
	\end{itemize}
	\subsection{Confidencialidad de Datos}
	\begin{itemize}
		\item Identificación de columnas con datos sensibles.
		\item Verificación de cifrado en reposo y en tránsito.
	\end{itemize}
	\subsection{Disponibilidad y Recuperación}
	\begin{itemize}
		\item Revisión de configuración de backups.
		\item Verificación de planes de restauración.
	\end{itemize}
	\subsection{Auditoría y Monitoreo}
	\begin{itemize}
		\item Comprobación de habilitación de logs de auditoría.
		\item Revisión de configuración de monitoreo de rendimiento.
	\end{itemize}
	
	\section{Riesgos Anticipados y Áreas de Enfoque}
	% --- (Esta sección se mantiene igual que la versión anterior) ---
	\begin{itemize}
		\item \textbf{Gestión de Privilegios:} Riesgo de permisos excesivos.
		\item \textbf{Protección de Datos Sensibles:} Posible almacenamiento en texto plano de datos de socios/clientes.
		\item \textbf{Consistencia de Datos:} Falta de integridad referencial.
		\item \textbf{Auditoría de Cambios:} Insuficiente trazabilidad.
	\end{itemize}
	
	\section{Próximos Pasos}
	% --- (Esta sección se mantiene igual que la versión anterior) ---
	\begin{enumerate}
		\item Seguimiento activo a la solicitud de accesos y documentación. \textbf{Dependencia crítica.}
		\item Reconocimiento detallado del esquema y configuración de la BD.
		\item Ejecución de pruebas planificadas (Gestión de Accesos primero).
		\item Documentación rigurosa de evidencia.
		\item Realización de entrevistas técnicas.
		\item Análisis de evidencia y redacción de hallazgos preliminares.
	\end{enumerate}
	$$ /int f(x) dx $$
	
	
	
\end{document}