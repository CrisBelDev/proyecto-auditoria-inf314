\documentclass[letterpaper,12pt]{article}

% --- PAQUETES ---
\usepackage[utf8]{inputenc}
\usepackage[spanish]{babel}
\usepackage{geometry}
\geometry{left=2.5cm, right=2.5cm, top=3cm, bottom=3cm}
\usepackage{setspace}
\usepackage[colorlinks=true, linkcolor=black, urlcolor=black, citecolor=black]{hyperref}
\usepackage{titlesec}
\usepackage{graphicx}
\usepackage{booktabs}
\usepackage{fancyhdr}
\usepackage{amsmath}
\usepackage{newtxtext,newtxmath} % Fuente Times New Roman
\usepackage{longtable} % Para tablas largas que ocupan más de una página
\usepackage{array}

% --- PAQUETES ADICIONALES PARA EVIDENCIA TÉCNICA ---
\usepackage{listings} % Para código SQL
\usepackage{xcolor}
\usepackage{float}    % Para posicionar imágenes

% Configuración de estilo para SQL
\definecolor{codegreen}{rgb}{0,0.6,0}
\definecolor{codegray}{rgb}{0.5,0.5,0.5}
\definecolor{backcolour}{rgb}{0.95,0.95,0.92}
\lstdefinestyle{mystyle}{
	backgroundcolor=\color{backcolour},   
	commentstyle=\color{codegreen},
	keywordstyle=\color{magenta},
	numberstyle=\tiny\color{codegray},
	basicstyle=\ttfamily\footnotesize,
	breakatwhitespace=false,         
	breaklines=true,                 
	captionpos=b,                    
	keepspaces=true,                 
	numbers=left,                    
	numbersep=5pt,                  
	showspaces=false,                
	showstringspaces=false,
	showtabs=false,                  
	tabsize=2,
	language=SQL
}
\lstset{style=mystyle}

% --- CONFIGURACIÓN DE ENCABEZADOS Y PIES ---
\pagestyle{fancy}
\fancyhf{}
\lhead{Auditoría de Base de Datos - FINDEPRO}
\rhead{Informe Final}
\rfoot{Página \thepage}
\renewcommand{\headrulewidth}{0.4pt}
\renewcommand{\footrulewidth}{0.4pt}

% --- CONFIGURACIÓN de TÍTULOS ---
\titleformat{\section}
{\normalfont\fontsize{14}{16}\bfseries}{\thesection}{1em}{}

\titleformat{\subsection}
{\normalfont\fontsize{12}{14}\bfseries}{\thesubsection}{1em}{}

% --- INICIO DEL DOCUMENTO ---
\begin{document}
	
	% --- PORTADA ---
	\begin{titlepage}
		\centering
		\setstretch{1.3}
		\vspace*{1cm}
		
		{\fontsize{16}{18}\selectfont\bfseries UNIVERSIDAD MAYOR DE SAN ANDRÉS \par}
		\vspace{0.2cm}
		{\fontsize{16}{18}\selectfont\bfseries FACULTAD DE CIENCIAS PURAS Y NATURALES \par}
		{\fontsize{16}{18}\selectfont\bfseries CARRERA DE INFORMÁTICA \par}
		
		\vspace{2cm}
		
		{\fontsize{16}{18}\selectfont\bfseries AUDITORÍA DE BASE DE DATOS \par}
		\vspace{0.3cm}
		{\fontsize{16}{18}\selectfont\bfseries INFORME DE AUDITORÍA \par}
		\vspace{0.3cm}
		{\fontsize{14}{16}\selectfont\bfseries Validación de Integridad del Sistema Integrado de Cartera (SIC) \par}
		\vspace{0.2cm}
		{\fontsize{14}{16}\selectfont\bfseries Entidad: FINDEPRO IFD \par}
		
		\vspace{2.5cm}
		
		{\fontsize{14}{16}\selectfont\bfseries Equipo Auditor: \par}
		\vspace{0.2cm}
		{\fontsize{12}{14}\selectfont
			Cristian Abel Mamani Mamani \par
			Carmen Reyna Candia Suñagua \par
			Gonzalo Rafael Esprella Coaquira \par
			Limbert Nelson Escobar Mamani \par
			Limberth Carbajal Colque \par
			Greis Estefani Gonzales Chávez \par
		}
		
		\vspace{1.5cm}
		
		{\fontsize{12}{14}\selectfont\bfseries Asignatura: Auditoría Informática (INF-314) \par}
		{\fontsize{12}{14}\selectfont\bfseries Docente: M. Sc. Miguel Cotaña Mier \par}
		
		\vfill
		
		{\fontsize{12}{14}\selectfont La Paz - Bolivia \\ Noviembre, 2025 \par}
	\end{titlepage}
	% --- FIN DE PORTADA ---
	
	
	
	\tableofcontents % Añade tabla de contenido
	\newpage
	\onehalfspacing % Interlineado 1.5
	
	% --- SECCIONES DEL INFORME ---
	
	\section{Introducción y Antecedentes}
	
	\subsection{Antecedentes de la Auditoría}
	El presente trabajo de auditoría surge en respuesta al Memorándum \textbf{GT-2025-104} emitido por la Gerencia Nacional de Tecnología de FINDEPRO el 3 de octubre de 2025. Dicha solicitud, bajo la referencia \textit{"Entrega de Diccionario de Datos (Req. Auditoría)"}, tiene como objetivo atender el requerimiento de Auditoría Interna AUD-2025-003 referente a la \textbf{"Validación de Integridad de la Cartera"}.
	
	La entidad se encuentra en un proceso de transición tecnológica donde coexisten registros de cartera propia (denominada 'FINDEPRO') y cartera comprada/recalificada, lo cual ha generado discrepancias en los reportes operativos y contables debido a limitaciones conocidas del Core Bancario.
	
	\subsection{Justificación}
	La integridad de los datos financieros es el activo más crítico para una Institución Financiera de Desarrollo (IFD). Errores en el cálculo de saldos, duplicidad de registros o inconsistencias en la mora pueden derivar en:
	\begin{itemize}
		\item Sanciones regulatorias por parte de la ASFI.
		\item Estados financieros erróneos que afecten la toma de decisiones.
		\item Pérdida de confianza de los socios e inversionistas.
	\end{itemize}
	Por tanto, esta auditoría informática es fundamental para asegurar la confiabilidad de la información que reside en el servidor \texttt{SRV-DB-01}.
	
	\section{Marco Teórico y Normativo}
	Para el desarrollo de la presente auditoría, se han considerado los siguientes marcos de referencia, estándares internacionales y normativa legal vigente en el Estado Plurinacional de Bolivia.
	
	\subsection{Marco Teórico Conceptual}
	
	\subsubsection{Auditoría de Base de Datos}
	Se define como el proceso sistemático, independiente y documentado para obtener evidencias y evaluarlas objetivamente con el fin de determinar en qué medida se cumplen los criterios de auditoría relacionados con la seguridad, integridad, disponibilidad y confidencialidad de los datos gestionados por un Sistema Gestor de Bases de Datos (SGBD).
	
	\subsubsection{Integridad de Datos}
	Refiere a la exactitud y consistencia de los datos almacenados a lo largo de su ciclo de vida. En el contexto de bases de datos relacionales, la integridad se rige por las propiedades ACID:
	\begin{itemize}
		\item \textbf{Atomicidad:} Las transacciones son todo o nada.
		\item \textbf{Consistencia:} La base de datos pasa de un estado válido a otro.
		\item \textbf{Aislamiento:} Las transacciones concurrentes no interfieren entre sí.
		\item \textbf{Durabilidad:} Los cambios confirmados son permanentes.
	\end{itemize}
	
	\subsubsection{Técnicas de Auditoría Asistidas por Computador (CAATs)}
	Son herramientas y técnicas que permiten al auditor utilizar la tecnología para realizar pruebas de auditoría. En este proyecto, se utilizan scripts SQL para realizar pruebas sustantivas de recálculo y validación de reglas de negocio sobre el total de la población de datos.
	
	\subsection{Marco Normativo y Estándares}
	
	\subsubsection{COBIT 2019 (Control Objectives for Information and Related Technologies)}
	Se ha seleccionado el dominio \textbf{DSS (Entregar, Dar Servicio y Soporte)}, específicamente:
	\begin{itemize}
		\item \textbf{DSS06 - Gestionar los Controles de los Procesos de Negocio:} Este objetivo asegura que la información procesada es íntegra, completa y válida. Es el criterio principal para validar que los saldos de cartera no han sido alterados ni duplicados indebidamente.
	\end{itemize}
	
	\subsubsection{Norma ISO/IEC 25012: Calidad de Datos}
	Esta norma define un modelo general de calidad de datos. Para esta auditoría, nos enfocamos en las características de:
	\begin{itemize}
		\item \textbf{Exactitud:} Grado en que los datos representan correctamente el verdadero valor del atributo (ej. saldos monetarios).
		\item \textbf{Consistencia:} Ausencia de contradicciones en los datos (ej. coherencia entre saldo vigente y estado de mora).
	\end{itemize}
	
	\subsubsection{Normativa Nacional (Bolivia)}
	\begin{itemize}
		\item \textbf{Recopilación de Normas para Servicios Financieros (RNSF) de la ASFI:} Específicamente el Libro 3, Título VII, Capítulo II sobre "Seguridad de la Información", que exige a las entidades financieras garantizar la integridad de sus bases de datos.
		\item \textbf{Ley N° 164 General de Telecomunicaciones:} En lo referente a la validez jurídica de los datos digitales y documentos electrónicos.
	\end{itemize}
	
	\section{Identificación de la Entidad y Entorno Tecnológico}
	
	\subsection{La Entidad: FINDEPRO IFD}
	FINDEPRO es una entidad regulada enfocada en el sector productivo. Su infraestructura tecnológica soporta operaciones críticas de microcrédito, vivienda y consumo.
	
	\subsection{Descripción del Sistema Auditado}
	El "Sistema Integrado de Cartera" (SIC) es el núcleo transaccional. La auditoría se centra en su backend.
	
	\subsubsection{Ficha Técnica de la Base de Datos}
	\begin{itemize}
		\item \textbf{SGBD:} Microsoft SQL Server 2016 Enterprise Edition.
		\item \textbf{Servidor:} SRV-DB-01.
		\item \textbf{Base de Datos:} FINDEPRO\_PROD.
		\item \textbf{Esquema:} [dbo].
		\item \textbf{Nivel de Acceso:} Lectura controlada para auditores externos.
	\end{itemize}
	
	\subsubsection{Estructura del Diccionario de Datos (Tabla Maestra)}
	Se ha relevado la estructura de la tabla principal \texttt{[Operaciones\_Cartera]}, objeto de las pruebas sustantivas:
	
	\begin{longtable}{|p{4cm}|p{3cm}|p{8cm}|}
		\hline
		\textbf{Campo} & \textbf{Tipo Dato} & \textbf{Descripción Técnica} \\
		\hline
		\endhead
		NUMERO\_DOCUMENTO & varchar(20) & Identificador único del cliente (CI). \\
		\hline
		SIGLA\_ENTIDAD & varchar(20) & Discriminador de origen ('FINDEPRO' o 'FAAA'). \\
		\hline
		NUMERO\_OPERACION & varchar(30) & Llave primaria compuesta (Suc-Prod-Sub-Of-Correlativo). \\
		\hline
		TIPO\_OBLIGADO & varchar(5) & \textbf{Campo Crítico.} '1A' (Titular), '2' (Garante). \\
		\hline
		MONEDA & char(3) & 'MN' (Bolivianos), 'ME' (Dólares). \\
		\hline
		SALDO\_VIGENTE & decimal(18,2) & Capital no vencido. \\
		\hline
		SALDO\_VENCIDO & decimal(18,2) & Capital en mora administrativa ($<=90$ días). \\
		\hline
		SALDO\_EJECUCION & decimal(18,2) & Capital en mora judicial. \\
		\hline
		DIAS\_MORA & int & Días de atraso calculados al cierre. \\
		\hline
		CALIFICACION & char(1) & Categoría de riesgo (A, B, C, D, E, F, H). \\
		\hline
		\caption{Estructura de la tabla auditada (Fuente: Diccionario de Datos TI-DOC-2025-089)}
	\end{longtable}
	
	\section{Objetivos y Alcance}
	
	\subsection{Objetivo General}
	Determinar la razonabilidad y confiabilidad de los saldos de cartera reportados por el sistema informático, validando la inexistencia de errores materiales debidos a duplicidad de registros o fallas en la migración de datos de la cartera 'FAAA'.
	
	\subsection{Objetivos Específicos}
	\begin{itemize}
		\item Verificar la consistencia entre los datos detallados y los reportes de control mayor (Contabilidad).
		\item Identificar registros que incumplan las reglas de integridad relacional (ej. operaciones sin titular o con múltiples titulares activos).
		\item Validar el cumplimiento de la lógica de negocio para la clasificación de la cartera (Vigente vs Vencida).
		\item Evaluar la calidad de los datos no financieros migrados desde sistemas legados (Excel).
	\end{itemize}
	
	\section{Metodología de Auditoría}
	
	La metodología empleada se basa en la ejecución de pruebas sustantivas mediante **Técnicas de Auditoría Asistidas por Computador (CAATs)**. El proceso se dividió en las siguientes fases:
	
	\subsection{Fase 1: Entendimiento y Relevamiento}
	Se analizó el documento \textit{"Diccionario de Datos - Sistema Integrado de Cartera"} para comprender la semántica de los datos y las reglas de negocio implícitas, como la distinción de titulares y garantes.
	
	\subsection{Fase 2: Adquisición de Datos (ETL)}
	Se procedió a la extracción de dos conjuntos de datos planos (CSV) del servidor de producción:
	\begin{enumerate}
		\item \texttt{Datos Origen.csv}: Cartera nativa.
		\item \texttt{OrigenRecalificado.csv}: Cartera migrada.
	\end{enumerate}
	Estos datos fueron importados a un entorno de pruebas SQL controlado por el equipo auditor para no afectar el rendimiento del servidor de producción.
	
	\subsection{Fase 3: Ejecución de Scripts de Validación}
	Se diseñaron y ejecutaron scripts SQL (\textit{Structured Query Language}) para:
	\begin{itemize}
		\item Unificar las fuentes de datos.
		\item Filtrar registros duplicados lógicos (Garantes).
		\item Recalcular sumatorias por moneda y estado.
	\end{itemize}
	
	\section{Ejecución y Hallazgos de Auditoría}
	
	En esta sección se detallan las pruebas realizadas, la evidencia obtenida y los hallazgos detectados, estructurados según las normas de auditoría.
	
	\subsection{Prueba Sustantiva 1: Conciliación de Saldos Globales}
	
	\textbf{Objetivo:} Verificar la exactitud matemática de los saldos de cartera.
	
	\textbf{Procedimiento:}
	Se ejecutó el script SQL \texttt{1\_4962977436586215840.sql} sobre la tabla unificada. La lógica aplicada fue sumar los campos \texttt{SALDO\_VIGENTE}, \texttt{SALDO\_VENCIDO} y \texttt{SALDO\_EJECUCION}, agrupando por \texttt{MONEDA}, y aplicando estrictamente el filtro \texttt{WHERE TIPO\_OBLIGADO = '1A'}.
	
	\begin{lstlisting}[caption=Extracto del Script de Auditoría]
		-- Validacion de Totales por Moneda (Extracto)
		SELECT 
		MONEDA,
		SUM(SALDO_VIGENTE + SALDO_VENCIDO + SALDO_EJECUCION) as Total_Cartera
		FROM Operaciones_Unificadas
		WHERE TIPO_OBLIGADO = '1A' -- Filtro critico para evitar duplicidad
		GROUP BY MONEDA;
	\end{lstlisting}
	
	\textbf{Resultados del Recálculo:}
	
	\begin{table}[H]
		\centering
		\begin{tabular}{lrrr}
			\toprule
			\textbf{Moneda} & \textbf{Saldo Sistema (Bs)} & \textbf{Saldo Control (Bs)} & \textbf{Diferencia} \\
			\midrule
			MN (Nacional)   & 11,987,441.48 & 11,987,441.48 & 0.00 \\
			ME (Extranjera) & 660,611.00    & 660,611.00    & 0.00 \\
			\midrule
			\textbf{Total}  & \textbf{12,648,052.48} & \textbf{12,648,052.48} & \textbf{0.00} \\
			\bottomrule
		\end{tabular}
		\caption{Matriz de Conciliación de Saldos}
	\end{table}
	
	\textbf{Conclusión de la Prueba:} Satisfactoria. No existen diferencias materiales. La integridad matemática de los saldos se mantiene intacta tras la migración.
	
	\subsection{Hallazgo 1: Riesgo de Integridad por Diseño Desnormalizado (Duplicidad)}
	
	\textbf{Condición:} 
	La tabla \texttt{Operaciones\_Cartera} presenta un diseño desnormalizado donde se almacena un registro por cada participante de la operación (Titular, Garante, Codeudor). Los campos de importes (\texttt{SALDO\_VIGENTE}, etc.) se repiten idénticos en cada registro asociado a la misma \texttt{NUMERO\_OPERACION}.
	
	\textbf{Criterio:} 
	\textit{COBIT 2019 DSS06.03 - Gestión de Roles y Responsabilidades:} El sistema debe garantizar que la información presentada a los usuarios minimice el riesgo de interpretaciones erróneas. 
	\textit{ISO/IEC 25012 - Característica de Eficiencia:} El almacenamiento redundante innecesario afecta el rendimiento y la consistencia.
	
	\textbf{Causa:} 
	Limitación técnica del Core Bancario legado que exporta vistas planas en lugar de relacionales para facilitar la lectura en Excel, sin considerar el impacto en bases de datos SQL.
	
	\textbf{Efecto (Riesgo):} 
	Alto. Si un analista de riesgos o un auditor externo realiza una consulta sumatoria (\texttt{SUM(SALDO)}) sin conocer la regla de negocio de filtrar por \texttt{'1A'}, el saldo de la deuda se duplicará o triplicará, generando informes financieros falsos y provisiones erróneas.
	
	\textbf{Evidencia Visual:}
	\begin{figure}[H]
		\centering
		\includegraphics[width=0.8\textwidth]{modelorelacional.jpg}
		\caption{Evidencia de registros duplicados para la misma operación (Titular y Garante)}
	\end{figure}
	
	\textbf{Recomendación:}
	\begin{enumerate}
		\item Crear Vistas (Views) de base de datos específicas para reportes gerenciales que filtren implícitamente \texttt{WHERE TIPO\_OBLIGADO = '1A'}.
		\item Restringir el acceso directo a la tabla base \texttt{Operaciones\_Cartera} y conceder permisos solo sobre las Vistas depuradas.
	\end{enumerate}
	
	\subsection{Hallazgo 2: Baja Calidad de Datos en Campos Descriptivos}
	
	\textbf{Condición:} 
	Se identificaron inconsistencias de formato en la columna \texttt{ACTIVIDAD ECON}. Varios registros presentan códigos CAEDEC con formato decimal (ej. "1111.0") o caracteres no numéricos, provenientes de la conversión automática de tipos de datos durante la importación desde Excel.
	
	\textbf{Criterio:} 
	\textit{ISO/IEC 25012 - Exactitud Sintáctica:} Los datos deben cumplir con el formato definido para su dominio.
	
	\textbf{Causa:} 
	Falta de validación de tipos de datos en el proceso ETL (Extracción, Transformación y Carga) al migrar desde hojas de cálculo.
	
	\textbf{Efecto (Riesgo):} 
	Medio. Aunque no afecta los saldos financieros, impide la correcta segmentación de la cartera por sector económico, lo cual es un requerimiento obligatorio para los reportes enviados a la ASFI.
	
	\textbf{Recomendación:}
	Ejecutar un script de limpieza de datos (\textit{Data Cleansing}) para estandarizar los códigos CAEDEC a formato texto de 5 dígitos.
	
	\section{Evaluación de Riesgos (Matriz)}
	
	Basado en los hallazgos, se presenta la matriz de riesgos residuales:
	
	\begin{table}[H]
		\centering
		\begin{tabular}{|m{4cm}|c|c|c|m{3cm}|}
			\hline
			\textbf{Riesgo Identificado} & \textbf{Probabilidad} & \textbf{Impacto} & \textbf{Nivel} & \textbf{Acción} \\
			\hline
			Sobreestimación de cartera por duplicidad de garantes & Alta & Alto & \textbf{Crítico} & Implementar Vistas SQL inmediatas. \\
			\hline
			Errores en reportes sectoriales por códigos CAEDEC & Media & Medio & \textbf{Moderado} & Depuración de datos planificada. \\
			\hline
			Diferencias en conciliación contable & Baja & Alto & \textbf{Bajo} & Riesgo mitigado (Prueba 1 exitosa). \\
			\hline
		\end{tabular}
		\caption{Matriz de Riesgos de la Base de Datos}
	\end{table}
	
	\section{Conclusiones Generales}
	
	El equipo de auditoría de la Universidad Mayor de San Andrés concluye que:
	
	\begin{enumerate}
		\item La base de datos \textbf{es íntegra en términos financieros}. Las pruebas de recálculo demostraron una coincidencia del 100\% con los saldos de control (Bs 12.6M en total).
		\item La migración de la cartera 'FAAA' se ha realizado correctamente a nivel de saldos, aunque hereda problemas de calidad de datos (formato) de los sistemas previos.
		\item Existe una debilidad de control interno significativa en la estructura de la tabla, que permite la lectura de datos duplicados si no se tiene conocimiento experto del modelo de datos.
	\end{enumerate}
	
	Se emite una opinión de \textbf{SALVEDAD SUJETA A MEJORAS}, recomendando la implementación inmediata de las vistas de base de datos sugeridas para mitigar el riesgo operativo.
	
	\vspace{2cm}
	\begin{center}
		\rule{8cm}{0.4pt} \\
		\textbf{Cristian Abel Mamani Mamani} \\
		Estudiante de Informática - UMSA
	\end{center}
	
\end{document}