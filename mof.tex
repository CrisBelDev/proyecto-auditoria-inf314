\documentclass[11pt, letterpaper]{article}

% --- PAQUETES ---
\usepackage[utf8]{inputenc}
\usepackage[spanish]{babel}
\usepackage{geometry}
\usepackage{graphicx}
\usepackage{fancyhdr}
\usepackage{titlesec}
\usepackage{booktabs}
\usepackage{array}
\usepackage{xcolor}
\usepackage{colortbl}
\usepackage{longtable}
\usepackage{enumitem}
\usepackage{helvet}
\usepackage{tikz} 
\usetikzlibrary{trees, shadows, positioning, shapes.geometric, arrows.meta}

% --- CONFIGURACIÓN DE PÁGINA ---
\geometry{left=2.5cm, right=2.5cm, top=3cm, bottom=3cm, headheight=50pt, footskip=1.5cm}
\renewcommand{\familydefault}{\sfdefault} 

% --- DEFINICIÓN DE COLORES ---
\definecolor{corporateblue}{RGB}{0, 51, 102}
\definecolor{tablegray}{RGB}{245, 245, 245}
\definecolor{headergray}{RGB}{225, 225, 225}

% --- CONFIGURACIÓN DE ENCABEZADOS Y PIES ---
\pagestyle{fancy}
\fancyhf{}
\lhead{
	\begin{tabular}{l}
		\textbf{\textcolor{corporateblue}{FINDEPRO IFD}} \\
		\scriptsize Gerencia Nacional de Tecnología
	\end{tabular}
}
\rhead{
	\begin{tabular}{r}
		\textbf{\scriptsize Manual de Organización y Funciones} \\
		\scriptsize Código: MOF-TI-V02
	\end{tabular}
}
\rfoot{\footnotesize \textit{Uso Interno - Copia Controlada} \hfill Página \thepage}
\renewcommand{\headrulewidth}{1.5pt}

% --- ESTILO DE TÍTULOS ---
\titleformat{\section}
{\color{corporateblue}\normalfont\Large\bfseries\uppercase}{\thesection}{1em}{}
\titleformat{\subsection}
{\color{corporateblue}\normalfont\large\bfseries}{\thesubsection}{1em}{}

% --- COMANDO PARA FICHAS DE CARGO (MODIFICADO: SIN NEWPAGE AL INICIO) ---
\newcommand{\fichacargo}[9]{
	\vspace{0.5cm} % Espacio antes de la tabla en lugar de página nueva
	\renewcommand{\arraystretch}{1.5} 
	\noindent
	\begin{longtable}{|p{16cm}|}
		\hline
		\rowcolor{corporateblue} \textcolor{white}{\textbf{\large DESCRIPCIÓN DE CARGO}} \\
		\hline
		\rowcolor{headergray} \textbf{1. IDENTIFICACIÓN} \\
		\hline
		\textbf{Denominación del Cargo:} #1 \\
		\textbf{Código:} #2 \\
		\textbf{Unidad Organizacional:} #3 \\
		\textbf{Dependencia Jerárquica:} #4 \\
		\textbf{Supervisión Directa:} #5 \\
		\hline
		\rowcolor{headergray} \textbf{2. PROPÓSITO DEL CARGO} \\
		\hline
		#6 \\
		\hline
		\rowcolor{headergray} \textbf{3. FUNCIONES Y RESPONSABILIDADES PRINCIPALES} \\
		\hline
		\begin{minipage}[t]{15.5cm}
			\vspace{0.2cm}
			#7
			\vspace{0.2cm}
		\end{minipage} \\
		\hline
		\rowcolor{headergray} \textbf{4. RELACIONES DE COORDINACIÓN} \\
		\hline
		\textbf{Internas:} #8 \\
		\textbf{Externas:} Proveedores de TI, Consultores, ASFI, Auditoría Externa. \\
		\hline
		\rowcolor{headergray} \textbf{5. PERFIL REQUERIDO} \\
		\hline
		\begin{minipage}[t]{15.5cm}
			\vspace{0.2cm}
			#9
			\vspace{0.2cm}
		\end{minipage} \\
		\hline
	\end{longtable}
}

\begin{document}
	
	% --- PORTADA ---
	\begin{titlepage}
		\centering
		\vspace*{1cm}
		\textbf{\Huge FINDEPRO IFD}\\[0.5cm]
		\textbf{\Large INSTITUCIÓN FINANCIERA DE DESARROLLO}\\[3cm]
		
		\colorbox{corporateblue}{\parbox{14cm}{\centering \vspace{1cm} \textcolor{white}{\textbf{\Huge MANUAL DE ORGANIZACIÓN Y FUNCIONES (MOF)}} \vspace{1cm}}}\\[2cm]
		
		{\Large \textbf{GERENCIA NACIONAL DE TECNOLOGÍA}}\\[0.5cm]
		{\large Versión Aprobada 2.0}\\[3cm]
		
		\renewcommand{\arraystretch}{1.5}
		\begin{tabular}{|p{5cm}|p{5cm}|p{5cm}|}
			\hline
			\textbf{Elaborado por:} & \textbf{Revisado por:} & \textbf{Aprobado por:} \\
			& & \\
			\vspace{1cm} & \vspace{1cm} & \vspace{1cm} \\
			Lic. Analista O\&M & Gerente Nal. Riesgos & Directorio General \\
			\hline
		\end{tabular}
		
		\vfill
		\textbf{La Paz - Bolivia} \\
		\textbf{Gestión 2025}
	\end{titlepage}
	
	\newpage
	
	% --- RESOLUCIÓN ---
	\section*{Resolución Administrativa}
	\addcontentsline{toc}{section}{Resolución de Aprobación}
	
	\noindent \textbf{RESOLUCIÓN DE DIRECTORIO N° 005/2025}
	
	\noindent La Paz, 15 de Enero de 2025
	
	\noindent \textbf{VISTOS:}
	
	El crecimiento operativo de la Institución Financiera de Desarrollo FINDEPRO; los requerimientos establecidos en la Recopilación de Normas para Servicios Financieros (RNSF) de la ASFI referentes a la gestión de riesgos tecnológicos; y el informe técnico TI-INF-2024-098 presentado por la Gerencia General.
	
	\noindent \textbf{CONSIDERANDO:}
	
	Que, es imprescindible fortalecer la estructura de la Gerencia Nacional de Tecnología para garantizar la integridad de los datos ante la migración de nuevas carteras de crédito.
	
	Que, se requiere segregar funciones críticas entre el desarrollo de software, la administración de infraestructura y la gestión de bases de datos.
	
	\noindent \textbf{POR TANTO:}
	
	El Directorio de FINDEPRO IFD, en uso de sus atribuciones estatutarias:
	
	\noindent \textbf{RESUELVE:}
	
	\textbf{PRIMERO.-} Aprobar la nueva estructura orgánica y el Manual de Organización y Funciones (MOF) de la Gerencia Nacional de Tecnología.
	
	\textbf{SEGUNDO.-} Instruir a la Unidad de Talento Humano la difusión del presente documento a todo el personal involucrado.
	
	\vspace{3cm}
	\begin{center}
		\begin{tabular}{ccc}
			\rule{6cm}{0.5pt} & \hspace{1cm} & \rule{6cm}{0.5pt} \\
			Presidente del Directorio & & Secretario de Directorio
		\end{tabular}
	\end{center}
	
	\newpage
	
	% --- CAPÍTULO 1 ---
	\section{Marco Organizacional}
	
	\subsection{Misión del Área}
	Proveer soluciones tecnológicas innovadoras, seguras y eficientes que soporten las operaciones financieras de FINDEPRO IFD, garantizando la disponibilidad de los servicios y la integridad de la información para la toma de decisiones estratégicas.
	
	\subsection{Organigrama Funcional}
	La estructura organizacional ha sido diseñada para cumplir con los principios de segregación de funciones exigidos por la normativa ASFI. A continuación se presenta la estructura vigente:
	
	\vspace{0.5cm}
	
	% --- INICIO ORGANIGRAMA TIKZ (AJUSTADO PARA NO DESBORDARSE) ---
	\begin{figure}[h]
		\centering
		% Resizebox fuerza a que el gráfico entre en el ancho del texto (textwidth)
		\resizebox{\textwidth}{!}{
			\begin{tikzpicture}[
				node distance=1.2cm,
				% Reducimos distancias internas por si acaso, aunque resizebox lo arregla
				level 1/.style={sibling distance=5.5cm},
				level 2/.style={sibling distance=3.5cm},
				level distance=2.5cm,
				% Estilo de las cajas
				box/.style={
					rectangle,
					draw=corporateblue,
					very thick,
					fill=white,
					text width=4cm,
					text centered,
					rounded corners=3pt,
					minimum height=1.5cm,
					drop shadow,
					font=\footnotesize\bfseries\sffamily
				},
				% Estilo de caja superior
				topbox/.style={
					box,
					fill=corporateblue,
					text=white
				},
				% Estilo de líneas
				line/.style={
					draw, 
					very thick, 
					-Latex, 
					corporateblue
				}
				]
				
				% Nodos
				\node[topbox, text width=6cm] (gerencia) {GERENCIA NACIONAL DE TECNOLOGÍA \\ (Ing. Carlos Méndez)};
				
				% Nivel 2: Jefaturas
				\node[box, below left=1.5cm and 2.5cm of gerencia] (desarrollo) {JEFATURA DE DESARROLLO \\ DE SOFTWARE};
				\node[box, below=1.5cm of gerencia] (dba) {ÁREA DE BASE DE DATOS \\ (DBA) \\ (J. Rodriguez)};
				\node[box, below right=1.5cm and 2.5cm of gerencia] (infra) {JEFATURA DE INFRAESTRUCTURA};
				
				% Nodos Staff (Seguridad y QA)
				\node[box, left=0.5cm of desarrollo, text width=2.5cm, fill=gray!10] (seguridad) {SEGURIDAD DE LA INFORMACIÓN (CISO)};
				\node[box, right=0.5cm of infra, text width=2.5cm, fill=gray!10] (qa) {CALIDAD Y PROCESOS (QA)};
				
				% Nivel 3: Operativo
				\node[box, below=1cm of desarrollo, text width=3.5cm] (analistas) {Analistas Programadores};
				\node[box, below=1cm of infra, text width=3.5cm] (soporte) {Mesa de Ayuda y Soporte};
				
				% Conexiones
				\path[line] (gerencia) -- (desarrollo);
				\path[line] (gerencia) -- (dba);
				\path[line] (gerencia) -- (infra);
				
				% Conexiones operativas
				\path[line] (desarrollo) -- (analistas);
				\path[line] (infra) -- (soporte);
				
				% Conexiones Staff (Líneas punteadas)
				\draw[dashed, thick, corporateblue] (gerencia.west) -| (seguridad.north);
				\draw[dashed, thick, corporateblue] (gerencia.east) -| (qa.north);
				
			\end{tikzpicture}
		} % Fin resizebox
		\caption{Estructura Orgánica - Gerencia Nacional de Tecnología}
	\end{figure}
	% --- FIN ORGANIGRAMA TIKZ ---
	
	\vspace{0.5cm}
	\noindent \small \textbf{Nota:} Las unidades de Seguridad de la Información y Calidad (QA) poseen reporte funcional a la Gerencia de Tecnología pero mantienen independencia de criterio reportando matricialmente al Comité de Riesgos.
	
	\newpage
	\section{Descripción de Cargos}
	% AQUI EMPIEZA LA PRIMERA FICHA SIN PAGEBREAK PREVIO
	
	% --- CARGO 1: GERENTE TI ---
	\fichacargo{Gerente Nacional de Tecnología}
	{TI-GER-001}
	{Gerencia Nacional de Tecnología}
	{Gerencia General}
	{Jefe de Desarrollo, DBA Senior, Jefe de Infraestructura, Oficial de Seguridad}
	{Planificar, dirigir y controlar la gestión integral de las Tecnologías de la Información y Comunicación (TIC), alineando la estrategia tecnológica con los objetivos comerciales de la IFD y asegurando el cumplimiento de la normativa regulatoria vigente.}
	{
		\begin{itemize}[leftmargin=*]
			\item Dirigir la elaboración y ejecución del Plan Estratégico de Tecnología (PETI).
			\item Aprobar las políticas y procedimientos para la gestión de cambios, incidentes y problemas en el entorno productivo.
			\item Autorizar pases a producción de nuevos desarrollos o modificaciones en el Core Bancario.
			\item Supervisar la gestión presupuestaria del área (CAPEX/OPEX).
			\item Representar a la entidad ante la ASFI en inspecciones tecnológicas.
		\end{itemize}
	}
	{Gerencias Nacionales (Operaciones, Riesgos, Finanzas, Auditoría).}
	{
		\begin{itemize}[leftmargin=*]
			\item \textbf{Formación:} Maestría en Dirección de TI o Finanzas.
			\item \textbf{Experiencia:} 7 años en cargos directivos en el sector financiero.
			\item \textbf{Certificaciones:} CISM, CGEIT (Deseable).
		\end{itemize}
	}
	
	% --- CARGO 2: DBA ---
	\newpage % SALTO DE PÁGINA MANUAL PARA LA SIGUIENTE FICHA
	\fichacargo{Administrador de Base de Datos (DBA) Senior}
	{TI-DBA-001}
	{Área de Administración de Base de Datos}
	{Gerencia Nacional de Tecnología}
	{Asistente de Base de Datos}
	{Garantizar la disponibilidad, integridad y seguridad de las bases de datos institucionales (SQL Server), administrando el ciclo de vida de los datos desde su diseño hasta su archivo histórico.}
	{
		\begin{itemize}[leftmargin=*]
			\item \textbf{Administración:} Instalar, configurar y mantener las instancias de base de datos en servidores de producción y contingencia.
			\item \textbf{Integridad:} Definir e implementar reglas de integridad referencial y restricciones (constraints) para asegurar la consistencia de los datos financieros.
			\item \textbf{Seguridad:} Gestionar usuarios, roles y privilegios. Realizar auditorías periódicas de accesos y trazas de base de datos.
			\item \textbf{Mantenimiento:} Ejecutar planes de mantenimiento (reindexación, actualización de estadísticas) para optimizar el rendimiento (Tuning).
			\item \textbf{Respaldo:} Ejecutar y monitorear la política de backups (Full, Diferencial, Transaccional) y realizar pruebas de restauración semestrales.
			\item \textbf{Soporte:} Proveer scripts y vistas personalizadas para requerimientos de Auditoría y Entes Reguladores.
		\end{itemize}
	}
	{Desarrollo, Infraestructura, Oficial de Seguridad.}
	{
		\begin{itemize}[leftmargin=*]
			\item \textbf{Formación:} Licenciatura en Informática o Sistemas.
			\item \textbf{Experiencia:} 4 años como DBA en entornos SQL Server.
			\item \textbf{Conocimientos:} T-SQL Avanzado, SSIS, SSRS, Alta Disponibilidad (AlwaysOn).
		\end{itemize}
	}
	
	% --- CARGO 3: JEFE DESARROLLO ---
	\newpage
	\fichacargo{Jefe de Desarrollo de Software}
	{TI-DEV-001}
	{Jefatura de Desarrollo}
	{Gerencia Nacional de Tecnología}
	{Analistas Programadores, QA (Quality Assurance)}
	{Liderar el equipo de ingeniería de software para el mantenimiento y evolución del Sistema Integrado de Cartera (SIC), asegurando la calidad del código y el cumplimiento de los requerimientos del negocio.}
	{
		\begin{itemize}[leftmargin=*]
			\item Planificar los sprints de desarrollo y asignar tareas al equipo de programación.
			\item Supervisar el ciclo de vida de desarrollo seguro (SDLC).
			\item Coordinar las pruebas de certificación de usuario (UAT) con las áreas de negocio.
			\item Gestionar el versionamiento del código fuente (Git) y los despliegues en ambientes no productivos.
		\end{itemize}
	}
	{Jefes de Producto, Riesgos, Usuarios Líderes.}
	{
		\begin{itemize}[leftmargin=*]
			\item \textbf{Formación:} Ingeniería de Sistemas o Software.
			\item \textbf{Experiencia:} 5 años en desarrollo de software (.NET/Java).
			\item \textbf{Conocimientos:} Scrum, DevOps, Arquitectura de Microservicios.
		\end{itemize}
	}
	
	% --- CARGO 4: JEFE INFRAESTRUCTURA ---
	\newpage
	\fichacargo{Jefe de Infraestructura y Comunicaciones}
	{TI-INF-001}
	{Jefatura de Infraestructura}
	{Gerencia Nacional de Tecnología}
	{Administrador de Redes, Soporte Técnico}
	{Asegurar la operatividad de la plataforma tecnológica (Data Center, Servidores, Redes y Comunicaciones), garantizando niveles de servicio adecuados para la operación de la entidad.}
	{
		\begin{itemize}[leftmargin=*]
			\item Administrar el Centro de Datos principal y alterno.
			\item Gestionar la virtualización de servidores y sistemas operativos (Windows Server/Linux).
			\item Monitorear la disponibilidad de enlaces de comunicación entre sucursales.
			\item Gestionar el inventario de activos tecnológicos de hardware y licencias.
		\end{itemize}
	}
	{Proveedores de ISP, Mantenimiento.}
	{
		\begin{itemize}[leftmargin=*]
			\item \textbf{Formación:} Ingeniería Electrónica o Telecomunicaciones.
			\item \textbf{Experiencia:} 4 años en administración de infraestructura crítica.
			\item \textbf{Conocimientos:} VMware, Cisco CCNA, Windows Server.
		\end{itemize}
	}
	
	% --- CARGO 5: ANALISTA DE SOPORTE ---
	\newpage
	\fichacargo{Analista de Soporte y Mesa de Ayuda}
	{TI-SOP-001}
	{Jefatura de Infraestructura}
	{Jefe de Infraestructura}
	{N/A}
	{Brindar asistencia técnica de primer nivel a los usuarios de la entidad, resolviendo incidentes relacionados con hardware, software ofimático y acceso a sistemas.}
	{
		\begin{itemize}[leftmargin=*]
			\item Atender y registrar los requerimientos de usuarios en el sistema de Mesa de Ayuda.
			\item Realizar mantenimiento preventivo y correctivo de equipos de computación.
			\item Gestionar las altas, bajas y modificaciones de usuarios en el Directorio Activo, previa autorización.
			\item Configurar impresoras, escáneres y dispositivos periféricos.
		\end{itemize}
	}
	{Todos los usuarios de la entidad.}
	{
		\begin{itemize}[leftmargin=*]
			\item \textbf{Formación:} Técnico Superior en Informática o ramas afines.
			\item \textbf{Experiencia:} 1 año en soporte técnico.
			\item \textbf{Conocimientos:} Windows 10/11, Office 365, Mantenimiento de PC.
		\end{itemize}
	}
	
	\newpage
	\section{Historial de Modificaciones}
	\renewcommand{\arraystretch}{1.3}
	\begin{longtable}{|p{2cm}|p{2.5cm}|p{7.5cm}|p{3cm}|}
		\hline
		\rowcolor{headergray} \textbf{Versión} & \textbf{Fecha} & \textbf{Descripción} & \textbf{Autor} \\
		\hline
		1.0 & 10/01/2018 & Emisión original por cambio de Core Bancario. & Gerencia TI \\
		\hline
		1.1 & 15/06/2020 & Inclusión de funciones de teletrabajo post-pandemia. & RRHH \\
		\hline
		2.0 & 15/01/2025 & Reestructuración por creación del Área de Seguridad y Ciberdefensa. & Ing. C. Méndez \\
		\hline
	\end{longtable}
	
\end{document}