%%%%%%%%%%%%%%%%%%%%%%%%%%%%%%%%%%%%%%%%%%%%%%%%%%%%%%%%%%%%%%%%%%%%%%%%%%%%%%%
% Este es el CÓDIGO LATEX para un Plan de Auditoría
% Basado en el estilo de '8Formula.pdf' y los datos de 'proyecto314.pdf'
%
% VERSIÓN CORREGIDA Y MEJORADA:
% 1. Corregido el error '\end{G}'
% 2. Añadido interlineado en tablas con '\arraystretch' (solicita del usuario)
% 3. Ajustadas columnas para evitar 'Overfull \hbox'
%%%%%%%%%%%%%%%%%%%%%%%%%%%%%%%%%%%%%%%%%%%%%%%%%%%%%%%%%%%%%%%%%%%%%%%%%%%%%%%

\documentclass[a4paper, 11pt]{article}

% --- PAQUETES ---
\usepackage[utf8]{inputenc}
\usepackage[T1]{fontenc}
\usepackage{lmodern} % Fuente LaTeX moderna
\usepackage[spanish]{babel} 
\usepackage[margin=2.5cm]{geometry} % Márgenes
\usepackage{parskip} % Espacio entre párrafos, sin sangría
\usepackage{array}   % Para \newcolumntype
\usepackage{longtable} % Para tablas que pueden ocupar varias páginas
\usepackage[table]{xcolor} % Para colores en tablas

% --- CONFIGURACIÓN DE TABLAS ---
% Para replicar el estilo del ejemplo (líneas y texto justificado)
\newcolumntype{L}[1]{>{\raggedright\arraybackslash}p{#1}}
\newcolumntype{R}[1]{>{\raggedleft\arraybackslash}p{#1}}
\newcolumntype{C}[1]{>{\centering\arraybackslash}p{#1}}

\definecolor{tableHeader}{gray}{0.9} % Color de cabecera

% ===>>> ¡AQUÍ ESTÁ LA LÍNEA PARA EL INTERLINEADO! <<<===
\renewcommand{\arraystretch}{1.5} % Aumenta el espacio vertical en las filas

% --- INICIO DEL DOCUMENTO ---
\begin{document}

\section*{PLAN DE AUDITORÍA}
\subsection*{INFORMACIÓN DE AUDITORÍA}

\noindent
\begin{tabular}{|L{5cm}|p{10cm}|}
	\hline
	\multicolumn{2}{|c|}{\cellcolor{tableHeader} \textbf{INFORMACIÓN DE LA ENTIDAD AUDITADA}} \\
	\hline
	\textbf{Empresa:} & FINDEPRO (Entidades Financieras de Desarrollo para Productores) \\
	\hline
	\textbf{Dirección:} & (Simulado) Av. Arce, Edificio Torres del Poeta, Piso 15, La Paz \\
	\hline
	\textbf{Representante de la empresa:} & (Simulado) Sra. Ana María Gutiérrez (Gerente General) \\
	\hline
	\textbf{Persona de contacto:} & (Simulado) Sr. Carlos Guachalla (Jefe de Sistemas) \\
	\hline
	\textbf{Teléfono:} & (Simulado) 22441122 \\
	\hline
	\textbf{E-mail:} & (Simulado) cguachalla@findepro.bo \\
	\hline
\end{tabular}

\vspace{0.5cm}

\noindent
\begin{longtable}{|L{5cm}|p{10cm}|}
	\hline
	\multicolumn{2}{|c|}{\cellcolor{tableHeader} \textbf{INFORMACIÓN DE LA AUDITORÍA}} \\
	\hline
	\textbf{Tipo de auditoría:} & Auditoría Académica (Proyecto de asignatura INF-314) \\
	\hline
	\textbf{Fecha de inicio de la Auditoría:} & (Sugerido) 27/10/2025 \\
	\hline
	\textbf{Fecha de fin de la Auditoría:} & (Sugerido) 08/12/2025 \\
	\hline
	\textbf{Norma objeto de la Auditoría:} & \textbf{Marco de referencia COBIT 2019} (con énfasis en dominios APO y DSS), \textbf{ISO/IEC 27001/27002}, y \textbf{Regulaciones ASFI} aplicables a IFD. \\
	\hline
	\textbf{Alcance auditado:} & La auditoría se centrará \textbf{exclusivamente en la base de datos principal} que soporta el Sistema de Gestión FINDEPRO, incluyendo:
	\begin{itemize}
		\item El servidor de base de datos (Configuración, versionamiento, parches).
		\item El esquema de la base de datos (Tablas, vistas, procedimientos, triggers).
		\item Los mecanismos de seguridad (Roles, permisos, cifrado).
		\item Los procesos de respaldo y recuperación (Backup \& Restore).
		\item Los logs de auditoría y rendimiento disponibles.
	\end{itemize}
	\textbf{No se incluye} en el alcance la auditoría de la aplicación cliente, la infraestructura de red o la seguridad física (excepto en lo que impacte directamente a la BD). \\
	\hline
	\textbf{Objetivo de la Auditoría:} & \textbf{Objetivo General:} Evaluar integralmente la gestión, seguridad y rendimiento de la base de datos de FINDEPRO, identificando riesgos, verificando la efectividad de los controles implementados y proponiendo recomendaciones para fortalecer la protección de los datos y optimizar su uso. \\
	\hline
	\textbf{Criterios de Auditoría:} & 
	\begin{itemize}
		\item Políticas internas de seguridad de FINDEPRO.
		\item Buenas prácticas para el SGBD específico (ej. CIS Benchmarks).
		\item Controles relevantes de ISO/IEC 27002 y COBIT DSS05/DSS06.
		\item Requerimientos normativos aplicables (ASFI, Protección de Datos).
	\end{itemize} \\
	\hline
	\textbf{Reunión de apertura:} & 
	\textbf{Fecha:} 27/10/2025 \quad \textbf{Hora:} 10:00 \quad \textbf{Sede:} (Simulado) Oficinas FINDEPRO \\
	\hline
	\textbf{Reunión de Cierre:} & 
	\textbf{Fecha:} 05/12/2025 \quad \textbf{Hora:} 10:00 \quad \textbf{Sede:} (Sugerido) Sala de Reuniones UMSA \\
	\hline
\end{longtable}

\subsection*{EQUIPO AUDITOR}
\noindent
\begin{tabular}{|L{5cm}|p{6cm}|L{4cm}|}
	\hline
	\multicolumn{3}{|c|}{\cellcolor{tableHeader} \textbf{MIEMBROS DEL EQUIPO}} \\
	\hline
	\textbf{Supervisor Académico:} & M. Sc. Miguel Cotaña Mier & (Docente INF-314) \\
	\hline
	\textbf{Líder Auditor:} & Cristian Abel Mamani Mamani & (Simulado) cmamanim@fcpn.edu.bo \\
	\hline
	\textbf{Auditor Junior:} & Carmen Reyna Candia Suñagua & (Simulado) ccandias@fcpn.edu.bo \\
	\hline
	\textbf{Auditor Junior:} & Gonzalo Rafael Esprella Coaquira & (Simulado) gesprellac@fcpn.edu.bo \\
	\hline
	\textbf{Auditor Junior:} & Limbert Nelson Escobar Mamani & (Simulado) lescobarm@fcpn.edu.bo \\
	\hline
	\textbf{Auditor Junior:} & Limberth Carbajal Colque & (Simulado) lcarbajalc@fcpn.edu.bo \\
	\hline
	\textbf{Auditor Junior:} & Greis Estefani Gonzales Chávez & (Simulado) ggonzalesc@fcpn.edu.bo \\
	\hline
\end{tabular}

\vspace{0.5cm}
\subsection*{FUNCIONES, RESPONSABILIDADES Y OBLIGACIONES}
\noindent
% Ajuste de p{0.48\textwidth} a p{0.47\textwidth} para evitar overfull box
\begin{tabular}{|p{0.47\textwidth}|p{0.47\textwidth}|}
	\hline
	\multicolumn{1}{|c|}{\cellcolor{tableHeader} \textbf{AUDITORES}} & \multicolumn{1}{c|}{\cellcolor{tableHeader} \textbf{AUDITADO}} \\
	\hline
	\textbf{FUNCIONES} & \textbf{FUNCIONES} \\
	\begin{itemize}
		\item \textbf{Líder Auditor:} Redactar, modificar y gestionar el plan de auditoría.
		\item Distribución del plan de auditoría.
		\item Recopilación de las evidencias necesarias.
		\item Detectar y verificar todos los hallazgos.
		\item Redactar y gestionar el Informe de la auditoría.
		\item \textbf{Auditor Junior:} Reforzar y aconsejar al equipo en temas técnicos.
	\end{itemize} & 
	\begin{itemize}
		\item Establecer los contactos y los horarios para las reuniones.
		\item Establecer la fecha de visita a los centros de trabajo.
		\item Ser testigos de la auditoría en nombre de la empresa auditada.
	\end{itemize} \\
	\hline
	\textbf{RESPONSABILIDADES} & \textbf{RESPONSABILIDADES} \\
	\begin{itemize}
		\item Comprender y dominar toda la materia auditada.
		\item Aclarar toda duda referente a la auditoría.
		\item Cumplir con el acuerdo de confidencialidad.
		\item Determinar la viabilidad de la auditoría.
		\item Gestionar de manera eficaz todo el proceso.
	\end{itemize} &
	\begin{itemize}
		\item Asistir a la reunión de apertura y cierre.
		\item Comprender, revisar y aceptar el Plan de auditoría.
		\item Suministrar al auditor todas las evidencias necesarias.
		\item Dotar al auditor del equipo de protección necesario (si aplica).
	\end{itemize} \\
	\hline
	\textbf{OBLIGACIONES} & \textbf{OBLIGACIONES} \\
	\begin{itemize}
		\item Conducta ética. Profesionalidad, integridad y discreción.
		\item Presentación ecuánime, informando y verificando con veracidad.
		\item Debido al cuidado profesional, con diligencia y buen juicio.
	\end{itemize} &
	\begin{itemize}
		\item Suministrar al auditor información veraz, precisa e inequívoca.
	\end{itemize} \\ % <--- ¡AQUÍ ESTABA EL ERROR '\end{G}', AHORA ESTÁ CORREGIDO!
\hline
\end{tabular}

\end{document}